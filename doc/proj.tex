% Example #1
% by LaTeX Rob

\documentclass[letterpaper]{amsart}
\usepackage{verbatim}
\usepackage{amsmath, amsthm, graphicx, url, pdfsync}
%\long\def\symbolfootnote[#1]#2{\begingroup\def\thefootnote{\fnsymbol{footnote}}
%\footnote[#1]{#2}\endgroup}
\author{Robert Franken}
\title{A hospital pharmacy patient information system}

\begin{document}
\maketitle
\part{Abstract}
\paragraph{} Hospital pharmacy operations primarily consist of the logistics of supply, budgeting and the provision of clinical services to wards, staff and patients. 
The focus of this project will be on part of these operations, namely ward pharmacy supply and clinical services, otherwise referred to as medicines management. In most hospitals, all records of processes carried out on wards by pharmacy staff are handwritten. The procedures followed are fairly robust, but slow and susceptible to errors of various kinds, with the information gathered not easily available to other staff members.
\part{Introduction}
\paragraph{}The system will enable pharmacy staff to use networked mobile devices and desktop terminals to document their ward activities. The system would be used to enter the details of patients, including their current ward, medication history and current medications, allergies, doctors, current levels of supply, planned date of discharge and potential clinical issues.  The entries would be immediately submitted to a central database that could be queried by pharmacy staff in other areas of the hospital and would be available to ward staff on subsequent ward visits.  In this way, some of the inefficiencies and risks inherent in the paper based recording system should be reduced.
The system will be implemented as a web application with web framework called Ruby on Rails.
\section{Background}
\paragraph{}Despite being extensive users of electronic information systems most hospitals still operate primarily with paper based recording and storing of clinical information.  The evidence for the improvements in patient care where electronic systems have been implemented is extensive.  Not only is there evidence of the predictable improvements in speed and reliability of storage and processing of information, but significant reductions in morbidity and mortality as well\cite{Amarasingham:2009xy}. 
\paragraph{}
Since the inauguration of the NHS Connecting for Health initiative in April 2005, the push to use technology in hospitals has increased and is a high priority for not only the United States, but also the NHS in the UK and worldwide.
%%%
An area identified by the Department of Health and the Audit Commission as being a priority for further development is Medicines Management - a broad term which refers to all aspects of the use of and logistics involved with medicines in hospitals and the community.\cite{spsugar} The term is also used within hospital pharmacy to refer to the part of the process that is carried out by pharmacy staff on wards.  This is where the focus of the project will be.
\section{Problem}
A technician or pharmacist typically visits a given ward daily and sees each patient on the ward to assess their pharmaceutical care and ensure that the patient and ward has sufficient supply of medications. During the visit, information about the patient is recorded on paper; one form to order medication supplies if required and another to record clinical information for reference by pharmacists (figure \ref{workflow}).  The medication order and clinical information is duplicated in an abbreviated form in a small section on a medication chart (a chart at the end of the patient's bed) in order to communicate with other pharmacy staff who may refer to the chart in the future.  On subsequent visits, the information is updated on both the patient's chart and in the pharmacy records.
Generally the system works well, however there are instances where there could be a time consuming duplication of effort when previously recorded information is not immediately available.  

\paragraph{ }A significant bottle-neck is the use of the patient's medication chart for communication with other pharmacy staff.  A medication chart records information about the patient's current and recent pharmaceutical treatment. When there are changes to the patient's treatment or space for further records is no longer available a new chart is written.  Any information regarding supplies or clinical information previously recorded on the original chart by pharmacy staff is filed away and not on the new chart.  Most ward pharmacy protocols require transcription of pharmacy information from the old chart to the new chart by pharmacy staff. This is a tedious and time-consuming process for which frequently there is not time and transcription errors can occur when it is done.  \\The result of this likely inavailability of information, information duplication, reliance on hand-written notes and time pressures is a relatively inefficient and unreliable and error-prone system. 
In other areas of public healthcare, simple errors in repetitive manual information tasks are responsible for a large proportion of errors leading to morbidity or mortality. There is evidence that supports the hypothesis that electronic recording and decision support systems reduce the rates of errors and improve patient outcomes.
\begin{figure}[]
    \centering
    \includegraphics[scale=0.25]{wardflow.bmp}
    \caption{An abstract view of a visit to a patient by pharmacy ward staff}
    \label{workflow}
\end{figure}
\section{Solution}
Problems with the current systems used in ward pharmacy medicines management relate to the lack of a single access point for medicines management information.  The aim of the project is to develop an application that may be used to record and access patient information and medication orders from a mobile platform as well as a standard desktop system.

\paragraph{}
\subsection{Requirements}
\subsubsection{Users} The users of the system will be pharmacy technicians, pharmacists that work on the ward, staff in the dispensary.  
\paragraph{ } Individual staff play several roles; pharmacists work on wards to resolve clinical and supply issues, as do some technicians. Nursing staff work closely with pharmacy staff to coordinate patient care, so giving nursing staff access to the system would facilitate asynchronous communication between the departments.  
\subsubsection{Data security} The primary concern for any information system project for a hospital is the confidentiality for patients.  This means compliance with the Data Protection Act 1998 and the recommendations of the Caldicot Report 1997.\cite{Caldicott}  Both of these reports state that the use of patient identifying information should be managed appropriately and that patient information should only be accessed where required. It can only be made available to individuals with a need for it and authority to access it. The Caldicott report and the World Health Organisation call for `best practices' to be observed in the development of health information systems\cite{whosec}.
\subsubsection{Medical records and law } All handwritten documents pertaining to patient care are required to be authenticated with a signature and are legal documents.  An electronic system will need to implement secure user authentication.  Consent to disclosure is usually obtained from a patient on admission to hospital. Given the greater accessibility of information stored in this system, a review of the disclosure agreement may be required.
\subsubsection{Interface} As the system is intended to replace parts of a existing functional system used by practiced users, speed, reliability and a clear logistic benefit are required.  In  figure \ref{workflow} there is a simplified model of the process that is followed for a visit to a patient in a ward.  What it does not indicate is that frequently the user may be required to execute the process for several patients in parallel.  The reasons for this are many, but are usually the result of a temporary unavailability of some element of the process.  This may be the patients themselves, or their medications may not be available at the time, or they have yet to be seen by the medical team.  The consequence is that when the resource does become available, the user needs to revisit an entry to complete the required information.  Time is often a factor as well, because the missing element may only be available for a short time (for example, a patient may have an appointment, or a doctor may present an opportunity by walking past the user).  Considering these examples, the system must allow input and access to data and processes in a direct and efficient manner. As users are likely to be interrupted frequently when inputting data and will frequently need to interrupt input in order to access other data, sessions must be able to retain data for immediate resumption. It is a requirement that partially completed data can be completed at a later stage.\\
\paragraph{ } The system needs to be usable by its target audience, but in the case of pharmacy staff, users who are typically highly technology literate, this aspect of the system will not require a high level of abstraction. Users are very likely to request additional functionality over time so the system needs to be as extensible and maintainable as possible. The system will also need an administration interface available to senior staff for managing users.  

\subsubsection{Hardware} Many hospital staff have in the last 5 years been equipped with HP windows PDA devices and before that PalmOS reigned supreme among technologically engaged medical staff.  These older devices are unlikely to be suitable for the job at hand, but they do indicate a trend that is likely to continue, as evidenced by the recent enthusiastic adoption of the Apple iPhone among medical staff\cite{skyscapeiphone}. 

\subsubsection{Scale} The scale of the system need not be great in terms of numbers of users.  It is unlikely that more than ten users will be using the system at any one time although this may change with additional features such as integration with laboratory results reporting or integration with the main pharmacy patient information database.

\subsection{Implementation}
There are a variety of capable mobile devices likely to be available to medical staff in the future.  It has been decided that the system should be as accessible as possible by implementing it as web application optimized for the iPhone, given its rate of adoption, capabilities and its availability for testing.
In order to satisfy data security and audit requirements, all persistent data should be stored on site and have authentication of users using the service and operations need to take place over an encrypted network.
As scale is not going to be an issue, the application could be run on a single server within the hospital's secure network.
From an operational point of view, medications management practices are thoroughly and consistently defined, so any electronic system that adheres to the practices already in place will meet the operational requirements for the system and not fall foul of data security problems. 

\part{Design}
This section describes the requirements of the system and how the requirements were used to design the system. By following an object oriented design approach with each real world entity being represented as a class, the design of the system may closely reflect the structure of the real world objects and the information expert gang of four pattern.
\section{Rails and MVC}
Except for small, simple web application projects, the MVC (Model View Controller) design approach should be used when developing web applications.\cite{tomcat}  The MVC model is used in place of a 2-tier, client/server model.  In the 2-tier model a majority or all of the business logic processing occurs on the client with the server providing the database.  This structure ties views and logic together on remote clients.  
Proposed in 1979 by Trygve Reenskaug when designing GUI systems at Xerox and widely used in OOP frameworks including that of the iPhone\cite{iphoneaction}. The MVC approach describes the `separation of concerns' within a system.  Related to the three-tier, or $n$-tier model, it breaks a system into several independent encapsulated modules\cite{rubesvid}.
\begin{figure}[]
    \centering
   \includegraphics[scale=0.5]{mvc_rubsvid.jpg}
    \caption{MVC structure}
\end{figure}

\paragraph{ }The MVC approach makes a system more modular and thereby more robust and amenable to maintenance and change.\cite{turborapid}
\\
\paragraph{ }The Model represents the problem domain of the system - the data and elements of the business logic (data validation for example)\cite{ibmsphere}. It is independent from the other parts of the system, using generic data formats for communication. Models may be considered observers of controller.  In the Rails approach, the model consists of the class representing an object and the database record holding its attributes. 
\paragraph{ }The Controller contains elements of the business logic of the system. The Controller is responsible for processing requests, forming requests to make to the Model and choosing which View mechanism to call. The Controller should be able to function with failure of the Model and View components\cite{ibmsphere}.\\ In Rails, the controller methods for a given model define the variables and direct which view is to be rendered.  The controller is where the view will be set to either iPhone or standard web interface is set. There may be multiple controllers for a single model.\\ 
\paragraph{ }The View contains all presentation information frequently being implemented with a template language. In most systems there are several Views available making it possible for the Control to choose the appropriate View for a given situation. Views may be considered observers of controllers and models.\\The view consists of the layout to be applied, either that for iPhone or standard, and the HTML passed to the layout.
\paragraph{ }
The encapsulation of elements of a system in this way provides several benefits over the 2-tier alternative.  By executing much of the processing of the data on a server before transmission, network activity is reduced.  Executing business logic on the server also allows for caching and reuse of results by multiple clients leading to greater overall efficiency.  Offloading processing from clients is of particular importance where mobile or lower power clients will be accessing the system.  The primary advantage of the MVC architecture is the benefit to design, development and maintenance that encapsulation provides.  By allowing for views to be managed separately from logic, different views for different clients may use results from the same logic, and by keeping related code together the points of interaction are kept to a minimum and as a result changes in each component are possible without affecting the others.\cite{unobajax}
\paragraph{ }  Within Rails applications there are further levels of encapsulation.  The classes representing each object in the model layer interact with the database via a compatibility layer provided by Rails called ActiveRecord.  ActiveRecord provides methods for database data manipulation and is database independent.  This allows for business logic code to function if the underlaying database system is changed, for example from MySQL to PostgreSQL.  Another example of further encapsulation in Rails applications is on the view side, with html views referencing calling partial views and then being passed to a layout, which calls CSS (cascading style sheets) and JavaScript libraries.
\section{Rails and REST}
Analogous to OOP and first proposed by one of the principle authors of the HTTP specification, Roy Fielding\cite{refac}, REST(representational state transfer) is the use of URLs to represent unique conceptual resources that are available to users in hypermedia systems.    An example of a REST resource may be \texttt{/patients/1} or \texttt{/discharges} - the former will always point to the patient with an id of 1, and that patient will never change, and the second will always point to patients who are being discharged.  The base routes follow the \texttt{:controller/:action/:id} format. The path \texttt{/patients/edit/1} passes the \texttt{:id =>1} to the edit action of the patients controller.  The representation of a resource may change but its underlying concept doesn't. \\
For each view required for a particular resource, for example for Patient, there is an associated model and action in the model's controller.  The system will define a views to index, new, show and edit patients and the patients controller will have an action for each of those, and there will be a separate ruby html file for each view.\\
The design of the system will follow the REST approach, focusing on conceptual resources.  
\section{Context and resources} 
The focus of this system is on the requirements of ward pharmacy staff.  Ward pharmacy staff work with information concerning wards, patients, doctors, prescriptions, products, orders, lab results, stores, and local policies.  These represent the REST resources that will guide the design.  The details of the these entities and how the information concerning them is detailed in this section.
\begin{figure}[]
    \centering
    \includegraphics[scale=0.5]{erd.jpg}
    \caption{Entity relationship diagram.  In the interests of clarity, models that do no more than facilitate a many-to-many relationship have been omitted}
    \label{fig 1}
\end{figure}
\subsection{Users}
There are three classes of user intended for this system.  Ward pharmacy staff, dispensary staff and ward nursing staff.  Each member of the ward pharmacy staff is either a pharmacist or ward technician.  Pharmacists and technicians have different roles, but the line between them continue to blur in the duties they execute.  The extent of this blurring depends on the site.  On traditional sites, technicians are only concerned with supply, whereas at increasingly more numerous sites pharmacy technicians carry out preliminary clinical checks and take drug histories from patients in addition to their supply functions.  While their roles are essentially identical in this system, a mechanism for changing access to functionality in the site will be made available.  Dispensary staff will simply use the system to view which patients are where, and who is being discharged during that day.  They will also require full access to orders generated for the wards and patients. \\ 
Pharmacist, Dispensary and Technician classes will all inherit from the User class. The design of this system will not differentiate in any functional way between pharmacist and technician users, but to do so in the future will be possible with this design.\\
The users model will also manage authentication control, in conjunction with user\_sessions and AuthLogic (discussed in implementation).\\
Attributes:
\begin{itemize}
    \item Username, encrypted password and persistence token - used for authorisation and session management.
    \item Firstname 
    \item Surname 
    \item Pager - used for contacting staff within the hospital.
    \item Band - band is a term used by NHS human resources indicating the seniority of the position, and is used within departments to assign responsibilities.  This attribute may be used to define authorisation for access to functionality in the future, for example setting all band 8b pharmacists to be have automatic administrative privileges. 
\end{itemize}
\subsection{Ward}
Wards are where patients stay for the majority period of each admission.  Each ward has a team of nurses and one or more regular medical teams associated with it.  All wards are assigned a patient group based on their current care requirements.
The following properties are relevant ward pharmacy staff:
\begin{itemize}
    \item Name - used to identify the ward
    \item Type - most wards care for patients admitted for specific problems, such as respiratory disease. 
    \item Pharmacist - a ward will have a single pharmacist who is responsible for pharmacy services to that ward 
    \item Ward notes - there is a variety of information about a ward and its workings a user may want to store to help other staff members who are charged with responsibility for the ward.  An example may be the times that are best to visit the ward, which beds hold the most critical patients, or who to speak to about a specific recurring issue. 
    \item Beds - a ward has several beds which may or may not be occupied. 
    \item Orders - wards have their own small stocks of medications that are supplied by the pharmacy department.  Ward staff assist in managing stock levels.
\end{itemize}
Ward pharmacy staff need to be able to list, edit and update ward orders, ward patients and ward notes. 
\subsection{Patient} The patients recorded in this system will be the focus of the system.  Most ward pharmacy functions centre on individual patients.  When working from a ward, pharmacy staff need to record and screen current medications, medication history, allergies, primary indications or diagnoses, supply levels, lab results and a brief medical history for each patient they see.
\subsubsection{Properties}
\begin{itemize}
    \item statutory details, including hospital identification
    \item current and previous prescriptions - and by extension, current pharmaceutical agents
    \item current and previous admissions
    \item allergies
    \item diagnoses and selected lab results 
    \item current and previous results of blood tests and values of other medical variables used to make clinical judgements. 
    \item notes other ward pharmacy staff have made about the patient 
    \item probable discharge date
\end{itemize}
Ward pharmacy staff need to view, list, update and add to these properties.

\subsection{Product}
A product may be an agent(drug) or combinations of agents, or other medical products such as dressings, delivery aids such as pill boxes or cutters.  
A product has a form, a strength and a variety of doses and frequencies that may be prescribed for it. 
\subsubsection{Product properties}
\begin{itemize}
    \item brand - the brand name of the product. 
    \item agents - combination products have multiple agents in them.  Pharmacists are usually more concerned with prescribed agents than what a specific brand is.  Only occasionally different brands have differences significant enough for the brand to be important. Each agent belongs to a pharmaceutical class of drugs. Much like classes in other areas of science, agents in the same pharmaceutical class share common characteristics.  For example, a patient may be allergic to beta-lactam antibiotics, of which one is penicillin.
	Classes also share common side effects and clinical indications, and the prescribing of multiple agents of the same class may be a prescribing oversight.  As such, being able to query the classes of product agents would be useful.
    \item form - a product may be have one of many forms, including several types of solid oral, liquid oral and topical preparations.  The dose of each form depends on this type - a liquid is dosed in volume, and a solid oral form is dosed in discrete units such as capsules.  To this end, a form has both a form type and a name to allow for calculations to be applied to them. 
    \item doses - how much of a product a patient is treated with is described by several variables.  These are the strength of each agent, e.g. 20 milligrams per unit, the number of product units, e.g. two tablets, and the frequency, e.g. twice a day or every eight hours.
    For a given strength, each product has a collection of valid quantity/frequency pairs associated with it.  A period of treatment must also be recorded.
    \item policies - a significant aspect of the potential benefit of recording information digitally is the prospect for some level of automation.  Part of the challenge of updating policies in a large organisation is disseminating them and encouraging compliance.  To this end, each product may have several policies.  Each policy is essentially text, intended to describe special cases regarding the use of the product.  These usually include conditions, such as 'do not use when taking aspirin'.  To assist in automation of some of the checks that are carried out by pharmacy staff, each policy has a number of checks associated with it.  Checks are entered by the user as a type of conditional statement.  Executing a policy's checks should result in some output indicating their results, either that all is fine or that something requires attention. 
    \item stores - each product is associated with one or more stores.  There are ward stores and a main pharmacy store.  Having access to current levels of a product in the pharmacy store while on a ward assists in early warning of potential supply problems.  A common problem is occurs when ordering prescribed items from the pharmacy store for a patient who is soon to leave.  If a product is in stock then all goes smoothly.  However, if levels are low or supply is entirely exhausted, this will not become apparent until it is too late to arrange additional supplies from external suppliers.  If when making an order the current level may be checked the orderer may immediately take action to prevent a break in supply continuity.  
\end{itemize}
Ward pharmacy staff need search products, view the details of products and prescribe products.  A product must be able to be compared with the patient to whom it is prescribed in order to identify potential problems.  Other details include stock level on the ward and in the pharmacy store.
\subsection{Prescription}
The recording of a prescription is this system will not be made by its prescriber but by ward pharmacy staff.  A prescription has the following properties or associations:
\begin{itemize}
    \item product - each prescription is for a single product object.
    \item dose - may be chosen from the available doses for a specific product.
    \item patient - a one-to-many relationship.
    \item prescriber - a one-to-many relationship, one doctor may have many prescriptions.
    \item treatment period - a start and end date is recorded, as is necessary with all prescriptions in hospitals, even if it is to be repeated indefinitely.
    \item ancillary instructions - text that gives additional dosing instructions beyond dosing instructions. 
    \item orders - for a given prescription there are usually multiple supplies in a hospital setting, often as the period of treatment may continue for some time.
    \item amount supplied - by recording the amount supplied following the completion of an order on the prescription, it will be possible to check where supplies are running low.  The quantities are recorded with two fields, packs and units.  The total may be inferred from this information, as the pack size will be recorded.  It is often useful to distinguish between a supply of one pack of 28 tablets and a supply of 16 tablets and 14 tablets.
\end{itemize}
Users will be able to create, update and view prescriptions for a specific patient, and create new orders for supply of the prescription.
\subsection{Prescription order}
Each prescription has multiple supplies made for it.  In existing systems, orders consist of orders for multiple prescriptions, or prescription order lines, each of which has the product and prescription details, but also instructions to the dispensary, such as priority for supply and whether or not instructions are required.
\begin{itemize}
    \item Packs - indicates how many full packs of the product are to be supplied. 
    \item Units - indicates how many individual units are requested. 
    \item IP - a boolean indicating if the patient is to have 'inpatient' labelling.  Inpatient labelling does not include instructions on the label, as the dose may be changed multiple times during admission which would require a change to the label. As dosing instructions are specified on the patient's drug chart, potentially conflicting instructions are not included on the label.
    \item Priority - this is a boolean indicating to the dispensary staff if the supply needs to be completed within 24 hours or the standard 3 days is sufficient. 
    \item Process delivered orders - a function to add delivered orders to the recorded supply level for the associated prescription.  In this way patient supply levels will be updated.  The function will also reflect this in the order status.
\end{itemize}
\subsection{Team}
A team, or more specifically a medical team, is a group of doctors.  Each team typically has two interns, a registrar and a specialist. 
\begin{itemize}
    \item Ward - the activity of a team typically centres on a single ward, though commonly cares for patients in more than one ward. 
    \item Patients - teams naturally have a collection of patients for whom they are responsible.
\end{itemize}
\subsection{Doctor}
\begin{itemize}
    \item Name 
    \item position - a doctor may have one of three positions, intern, registrar or specialist.  Within those positions are grades, but for this system, those three roles are all that is relevant to pharmacy staff. 
    \item page number - the need to communicate with prescribers means that page numbers are required. 
    \item interns, registrar and specialist - functions to return doctors with specific roles.
    \item specialty - a team is defined by its leader, the doctor with the 'specialist' role.  So, a team's speciality will be found by querying its specialist's specialty.
\end{itemize}
\subsection{Store}
A store is associated with many products and with a single ward or with the pharmacy department.  For the purposes of this system, pharmacy stores are identical to ward stores, except the value in the ward foreign key will be nil.  It would be possible to use a polymorphic relationship, similar to that used for policies, but this would introduce complexity without utility.  Each store has a current store level for each product and an order for each product.  This means that connecting models are required that are more complex than those for the other many-to-many relationships in this system.
\begin{itemize} 
    \item shelf - conceptually, each shelf is in a specific store, holds a specific product and has a current level.  In this system, conceptual shelves are represented by records in product\_stores.  Stores and products have a many-to-many relationship via product\_stores, each product\_stores record contains its foreign keys for both store and product, and a stock level for that product.
    \item store orders - each store has multiple store orders.
\end{itemize}
\subsection{Store order}
A store order is an order to the pharmacy for a top up of the ward's stock.
\begin{itemize}
    \item store - each store order is associated with a single store.
    \item store order line - product\_stores share a many-to-many relationship with store\_orders through store\_order\_lines.
    \item has many store order lines, each of which belongs to a specific product in a specific store, a product-store line.
\end{itemize}
\subsection{Level}
Most patients when admitted to hospital will have multiple tests carried out during their admission.  Most of these will be from blood samples, and a majority of those will be repeated for every sample received.  This results in something similar to a set of collections of the tests for each patient, each collection having a time of collection or measurement.  A 'level' in the system represents a collection of results of tests for a given time.  Each patient has a collection of levels.
\subsection{Policy}
As described earlier, a policy reflects some of the checks carried out by pharmacy staff.  Most of the checks pharmacy staff carry out are standard throughout the profession, but there are local policies in each hospital that are very specific and not obvious.  An example of a local policy may be a product that may only be prescribed by members of a specific medical team, or a brand of drug that is peculiar to a specific indication.
\begin{itemize}
    \item Content - this is plain text that is intended to be used to describe the policy. 
    \item Checks - a policy may have many checks associated with it.  These checks are conditional statements that may be executed entered by an administrator and executed against a patient.
\end{itemize}
\subsection{Check}
A check is a collection of conditional statement elements which are added to a policy by an administrator used to generate executable checks against a given patient.  What is entered by the user is not code, but something that may be used to create to necessary code for the checks.  The entries will represent the main elements of the generated conditional statement.
\subsection{Admission} An admission is the period between an admission and discharge of a patient, as well as the reason (indication) the patient has been admitted.  Usually a single medical team is responsible for the treatment of a patient at any one time.  During an admission, a patient may be transferred between teams, but the final team for an admission may be considered the primary team.  Having access to some basic information regarding previous admissions easily assists in finding pertinent information concerning the patient quickly for future admissions.   
Properties of an admission:\\
\begin{itemize}
    \item date of admission 
    \item ward admitted to 
    \item discharge (departure) date - this is useful even for current admissions.  An expected departure date assists in discharge planning and ensuring sufficient supplies of medications are supplied in a timely fashion. 
    \item primary indication (reason for admission) 
    \item summary of clinical information regarding that admission 
    \item a function to find all patients being discharged in range of days.
\end{itemize}
\section{User interface}
The user interface for mobile devices and for terminals will follow the same RESTful structure, focused on the conceptual resources of the system.  Rather than display a large amount of information on a single page or screen, the interface will consist of a greater number of simple pages with a clear navigation scheme in a tree-like structure.  An example of the result of a REST driven interface is the navigation path followed to find a patient's current medications.  Once logged in, a ward pharmacy staff user will be directed to a list of wards with some navigation links at the bottom of the view.  One of these links will be to all patients.  Following this link will display a list of all patients registered in the system with the option to filter the list with a search option.  Each entry of Patient will list relevant patient attributes, with navigation links for each patient.  Following the link to display a specific patient leads to the display of that patient's details including current admission details and the results of policy checks.  It is from this view that lists of admissions notes, admissions and prescriptions may be retrieved, as well as creation of new prescriptions and navigation back to the ward.  There will be some exceptions to this as dictated by some use cases, such as a link to a list of patients being discharged on the current day from the first page after login, and inclusion of most recent admission information in a patient view, or most recently delivered orders for a ward's store in the ward view.\\
\begin{figure}[]
    \centering
   \includegraphics[scale=0.5]{uiflow.jpg}
   \caption{A simplified navigation tree.  For pharmacists the starting point after login would be a list of wards(Ward), for dispensary staff, a list of patients being discharged (Patient).}\label{uiflow}
\end{figure}
The pattern of navigation described will be consistent for each resource recorded in the system.  The process followed create, view, edit and delete products will be similar to that for patients or prescriptions.  The variation will come with the decisions to either render a collection of associated objects in a separate view, or as part of the view for its 'owner', such as with agents of a product, or admissions of a patient.  By constructing the user interface in this way it is the hope that it will be painless for users to understand and navigate.\\
\section{Users}
\begin{figure}[]
    \centering
   \includegraphics[scale=0.5]{usecase.jpg}
   \caption{Use case diagram for three classes of user.  CRUD (Create, Read, Update and Delete) actions have been grouped in aid of clarity.  In this system, pharmacist and technician roles are treated as equivalent. }\label{usecase}
\end{figure}

\subsection{Pharmacists}
\begin{itemize}
    \item view all wards -a list of all wards registered on the system with ability to navigate to a more detailed view of the system.
    \item view details of a  ward.
    \item a list of the patients on a specific ward.
    \item the details of each patient on the ward.
    \item the admissions associated with each patient.
    \item currently and previously prescribed products for each patient.
    \item orders for supply of a prescription.
    \item orders for a ward supply.
    \item details of a product.
    \item ward medical team.
    \item search for patients or products. 
\end{itemize}
Other operations:
The main function of a ward visit by a pharmacist, and to a lesser extend a ward technician, is recording and checking each patient's medical condition and treatments.  Checks are made in reference to professional knowledge, and to local policies, such as availability of a given brand of a medication.
\begin{itemize}
    \item Check prescribed treatments against medical condition.
    \item Check prescribed treatments against one another.
    \item Check current stock levels in ward stores.
    \item Check patient medication supplies.
    \item Order medications when supplies reach a given level.
\end{itemize}

\subsection{Dispensary staff}
Dispensary users need to interact with the ward and patients concerning supply issues.  As such, dispensary pharmacy staff need to be able to view following:
\begin{itemize} 
    \item Discharges.
    \item Patient current prescriptions.
    \item Patient supply orders.
    \item Store orders.
    \item Ward details. 
\end{itemize}

\subsection{Administrator}
An administrator will need to administer users, wards and products.  To this end, user with administrator privileges will need to perform CRUD operations of the following:
\begin{itemize}
    \item Products, agents and doses.
    \item Wards, including stores and product stores.
    \item Doctors and teams.
    \item Users
    \item Policies and checks.
\end{itemize}

\part{Implementation}
The structure of rails applications closely mirror that of the underlying database which allows for straight forward translation of classes and instances to and from the database by ActiveRecord, and mapping of classes to their CRUD operations.   This structure
The Rails framework convention provides CRUD methods for objects and some associations when objects and their associations are created.  For example, when creating a new model with the Rails model generation script, a standard Rails class is represented in the database by a single table witch is named the plural of the class name.  In the case of the Patient class, the associated table is called 'patients'.  The existing class objects' attributes are represented by a row in their class' table.  With this standard structure, methods for finding all patients, editing patients, creating new patients and deleting patients are provided, along with views for these actions.  The views provide basic forms with which to update.  These provide the basis for extension to show and manipulate the data of instances of multiple associated models, and providing tailored views. 
\section{Ruby on Rails features}
\subsection{ActiveRecord}
ActiveRecord is a Rails library which provides an ORM (object relational mapping) layer to database interaction.  This allows for a convergence of the database and the business logic into a single entity.  For example, \texttt{Admission.leaving(1)} to retrieve a collection of patients that are being discharged between now and tomorrow with the static method \texttt{leaving(arg, *arg)}, and for all patients, rather than writing an SQL query, a static method may be called - \texttt{Patient.all} - which generates the required SQL. Similar ActiveRecord methods are Patient.first, Patient.last.  These may be added to with named scopes, allowing for calls like Patient.first.admissions.current.  The named scope is defined in the Admission model \texttt{named\_scope :current, :conditions =>  ["admdate <  NOW() AND depdate > NOW()"]}.  In addition to implementing the ORM patter, ActiveRecord also implements single table inheritance and associations, allowing for potentially potentially complex SQL queries to be generated in ways that are straight forward to read, for example \texttt{Ward.find(3).patients} would return a collection of patients for the ward with and \texttt{id} of 3.\\ \texttt{Patient.first.admissions.last.ward.pharmacist} would return a Pharmacist object for the pharmacist who was responsible for the first patient during their last admission.  The pharmacist is a type of user, but may be called directly due to STI (single table inheritance), where the class of the User is recorded in a 'type' attribute of the Users table.  Where ActiveRecord fails to provide what is desired, it is possible to write methods that return the collection you may want and then call a standard ActiveRecord method on the result.\\ for example \texttt{Patient.first.policy\_results.first[:product].description} uses a combination of standard ActiveRecord and custom instance methods in a readable and concise syntax.
\subsection{External libraries}
There are many Rails plugins of varying standard generated by the Rails community.  I have utilised three, AuthLogic for user authentication, thinking-sphinx to generate search results when searching for products or patients, and iUI for the iPhone user interface. Thinking-sphinx is a very popular plugin and may be considered a capable and stable product, whereas iUI was written two years ago as a personal project and has had little maintenance since. iUI essentially consists of CSS and JavaScript which respond to tags in the HTML.  This works but is difficult to implement when there are multiple views required for the same page, as tags that are required for the iPhone rendering with iUI may not be appropriate in the standard web view.
\subsection{Generator scripts}
Rails provides generation scripts allowing for the creation of new models, controllers, scaffolding or a new project with a single line at the command line.  To create this application, the first generation script used was rails:\\
\texttt{rails pharm}, then a scaffold was created:\\ \texttt{ruby script/generate scaffold patient FirstName:string MiddleName:string Surname:string dob:date allergies:text sensitivities:text}. \\
A scaffold generation results in the generation of a controller with default actions defined, an empty model, views for edit, index, view and show actions, and a database migration that when run creates a table in the database specified in the database.yml configuration file.   
Alterations to an existing model are most easily executed with the migration generation script.  Migrations are timestamped and when executed, alter database schema configuration file and the database itself.  Migrations are stored to facilitate application migration - if migrations are written correctly then is possible to implement all of them in a new environment with a single command, and to go back to a previous state.
The migrations used most frequently in this project were to add, remove or rename attributes in models.  
\subsection{Routing and REST}
A central tenet of Rails applications is the that they should be 'RESTful'.  REST organisation is enforced by the scripted scaffolding and default routing.
One of the ways in which Rails leverages a REST architecture is in routes.  The routes file enables the nomination of controllers as REST resources with associated paths.  An example of this is the route \texttt{/discharges}.  The route is defined in the routes.rb file with \texttt{map.resources :discharges} which references the \texttt{discharges\_controller.rb} controller.  In the controller, the only action defined is \texttt{index} where the collection of patients to be indexed is defined, in this case with a function \texttt{leaving(1)}.  Routes also assist in defining nested associations, such as \texttt{map.resources :products, :has\_many => :agents}.  Assuming the correct associations are defined in the relevant models for Product and Agent, and there is a connecting model in the case of a many-to-many relationship, this route maps the \texttt{product/:product\_id/agents} providing a meaningful path.
  
\section{Polymorphism and inheritance}
Much of the polymorphism employed is provided by the rails scaffolding in the rails base classes.  However, the store object could be a pharmacy store or a ward store and they needed to be differentiated.  Rather than have a separate attribute for each kind of association, the convention in rails at the database layer is to use two attributes to identify what kind of store it is, a single foreign key attribute that stores the primary key of all off the associations and a type attribute to indicate the class of the association.  Both are later used in the model layer to define associations for ActiveRecord.
%TODO discuss indicationable structure for agents, products and classes


\section{Interface}
Multiple layouts are required to cater for system administration from a desktop or standard screen and for mobile users.  The system chosen for defining the view uses the user agent reported by the client in the application controller and sets the format for the respond\_to methods in the various controllers.  If it is a mobile safari client the iPhone views are rendered with the iPhone layout.  
\subsection{Standard web interface}
The interface intended for use on terminals with standard size displays - that is, greater than the 480 x 320 pixel\cite{iphonespecs} of an iPhone or similar device. Most wards have at least one terminal available, and it is likely that inputing larger quantities of information will be easier from a terminal.  Terminals will also be used by dispensary and nursing staff.
\subsection{Mobile interface}
The mobile interface for the system does not deviate to any significant degree from the standard interface in terms of navigational flow.  The primary difference is in presentation through the use of separate templates and layout, and an external user interface framework, iUI\cite{iui}, which facilitates the construction of user interfaces analogous to those of native applications.
\section{Classes and methods}
\subsection{Users and sessions} 
The three primary roles reflected in the system are those of technicians/pharmacists, dispensary staff, and administrators.  Any administrator will also be a technician or a pharmacist, so a boolean property has been used to define users with administration rights.  Rails and the rails plug-in AuthLogic create a session cookie (user\_credentials) upon user login.  By this mechanism the current session and user is persisted.  Data associated with a session may be stored in the session hash but will be lost if the user logs out, closes the browser or clears their cookies.  As a result, storing data objects in a session is not advisable.  Doing so would also lead to stale data in the object or the database.\\
As the system will have a limited number of users and roles, a simple access restriction pattern is appropriate.  In this case a boolean property, admin, is employed to allow access to administration functions.  The model controllers check the property on instantiation so that both the controller and associated views have access to the admin attribute of the user.  In combination with a check of the current\_user and user associated with records, editing of users and other records are restricted.\\
When a user logs into the system, they are directed to a view of a list of wards in they are a pharmacist, or a list of patients to be discharged on that day if they are a dispensary user.  If there are no discharges for the day, the ward list is shown instead with a notice that there are no patients being discharged on that day.  These behaviours are defined in the user\_sessions\_controller.rb, where user class is tested and the routing defined as appropriate.  The session remains if the browser is closed without a logout, with the option of a timeout, which allows for the multiple brief access periods likely in a mobile setting without a need to login each time.
\subsubsection{Table description}
User(id: integer, username: string, crypted\_password: string, password\_salt: string, persistence\_token: string, created\_at: datetime, updated\_at: datetime, admin: boolean, type: string, firstname: string, surname: string, pager: string, band: string)
\subsection{Ward}

\subsubsection{Table description}
Ward(id: integer, name: string, specialty: string, nurse: string, ward\_notes: text, created\_at: datetime, updated\_at: datetime, pharmacist\_id: integer)
\subsection{Check}
In order to interpret the checks entered by a user a three hashes are used to map the user choices, which are made available by a menu, to actual method calls.  The structure of implementation means that in order to add additional potential conditional tests all that is required is to enter the appropriate values in the hash.  The matching values are method calls that are called on the patient object, for example:\\
\begin{verbatim}
    @@type_hash = { "categories" => ["current_agents_categories_names"], 
	    "agents" => ["current_agents_names"], "level" => ["levels", "last"], 
	    "specialist" => ["team", "specialist", "first"], 
	    "specialty" => ["team", "specialty"] }  

    @@check_hash = { "bp" => "dbp", "k" => "potassium", "surname" => 
	    "surname"}

    @@operator_hash = { ">" => ">", "<" => "<" , "=" => "==", "has" => 
	    "include?" }

    @patient.send (@@type_hash["level"].send(@@check_hash["bp"]
	    .send(@@operator_hash[">"], value)

\end{verbatim}
would result in, where value is 90:\\
\begin{verbatim}
    @patient.levels.last.dbp.>(90) #=> boolean

\end{verbatim}
The actual code is slightly more complex, but this example illustrates the mechanics.  The methods called in the result are all generated when the models are created.  This arrangement allows for checks that have not been defined in the code itself.  Allowing the user to execute code directly is potentially dangerous, but using a hash to define the conditions prevents access to methods other than those deemed appropriate.

\subsubsection{Attributes}
\begin{itemize}
    \item check\_type - essentially the first function to call on the patient instance.
    \item check - second function, may be nil as some potential checks may only require a single method call, for example \texttt{@patient.surname == 'Smith'} only needs \texttt{check\_type}
    \item operator - indicates the final method, which is passed the value with the ruby method \texttt{send(String method, value)}.
    \item value - a string, integer or float that is passed to the 'operator'.
\end{itemize}

\subsubsection{Methods}
\begin{itemize}
    \item perform\_check(patient) - passes check attributes to private function (arrayit) which returns an array of function calls, which is then passed to a second private function (process) that recursively calls each element of the array, calling the last with the value to be passed.  The result, which is returned, is returned as a hash containing the check itself and the boolean result of the check. 
    \item description - returns a string representing the check.
\end{itemize}

\subsubsection{Table description}
Check(id: integer, policy\_id: integer, check\_type: string, check: string, operator: string, value: string, created\_at: datetime, updated\_at: datetime, test: boolean)

\subsection{Patient}
\subsubsection{Patient methods}
\begin{itemize}
    \item current\_products - returns the products the patient is currently prescribed. 
    \item current\_agents  - each product may consist of more than one agent, and the agents are what clinical staff are most concerned with.
    \item current\_agents\_names - returns a collection of strings representing current agents.  Allows for simpler listing of current agents. 
    \item current\_agents\_categories\_names - returns a collection of strings representing the categories, or pharmaceutical classes, of the agents that the patient is currently prescribed.  This information is useful in performing checks on the current treatment of the patient, as some pharmaceutical classes are contraindicated (should not be prescribed) in some situations. 
    \item current\_level - returns the last set of results for a patient.  Most are lab results, but may be others, such as diastolic blood pressure, weight or water input/output balance. 
    \item has\_currentnotes? - returns boolean indicating whether the current admission for this patient has notes associated with it. 
    \item inpatient? - returns boolean indicating whether the patient is currently admitted or not. 
    \item current\_ward - if currently admitted returns an instance of ward representing ward the patient is currently admitted to, otherwise returns nil. 
    \item fullname - returns a string representing the full name of the patient.  Used in views. 
    \item check\_allergies - returns a set of strings resulting from the intersection of a set of agents representing the patient's allergies and a set of currently prescribed agents.
    \item team - returns instance of team that represents the medical team currently responsible for the patient's care. 
    \item policy\_results - returns an array of hashes, each hash containing a product and an array of check results. 
    \item failed\_policies - returns an array of hashes, each hash containing a product and an array of the checks that failed to pass. 
    \item print\_failed - prints a description of failed checks 
    \item CRUD methods
\end{itemize}
\subsection{Level}
Each individual level type, such as potassium or creatinine (kidney function), is represented as a column in the levels table.  In order to add tests, a rails migration would be required.  All that would be required to add magnesium to the tests would be: \texttt{script/generate migration Add\_Magnesium\_To\_Levels magnesium:float}.
\subsubsection{Table description}
\texttt{Level(id: integer, potassium: float, creatinine: float, inr: float, ppt: float, albumin: float, created\_at: datetime, updated\_at: datetime, patient\_id: integer (foreign key), dbp: integer, collected: datetime)}

\part{Discussion}
\section{Outcome}
\section{Experience}
\section{Improvements}
\subsection{Testing} 
Given the intended realm of use for the system, issues of security speed and adherence to the multitudinous local and national policies, guidelines, regulations and laws specifying the appropriate handling of personal medical information would need careful consideration.  As it stands, the security of the system would be relatively assured if the network used for access to the patient data were provided over encrypted wireless networks and local policies prescribed sensible behaviours for its users.
\subsection{Authentication}

\subsection{Integration}
Data duplication is a already existing problem in most hospital data, so a new system such as this would be of little use unless it was integrated into existing systems.  As most are almost entirely proprietary integration would likely require cooperation from the suppliers of existing systems and present some technical challenge.
\subsection{Interface}
There are many possible variations with regards to the finer points of the implementation of the interface.  Many of the decisions are based on a specific work flow - something that can vary from site to site.  The views currently provided are on the most part simple lists or collections centred on the resources recorded.  If the system were to be deployed on a site, part of the testing and integration would likely involve user requests for changes or customisations in navigation or views, such as a more detailed search facility or links to custom views from specific points.  Changes and refinements of that nature would be relatively painless to implement due to the structured implementation of the application.\\
It is likely that a variety of mobile devices capable of using the system would be available.  The design allows for multiple views and layouts.  Its possible that simply rendering existing views with a different layout may suffice in these cases, in which case all that would be needed would be an additional layout file.  The interface as it stands is functional, but rudimentary.  Judicious use of AJAX technologies in rendering and editing collections, which is a large part of the functionality of the system, would provide a more responsive system despite the increase in complexity.
\subsection{Refactoring}
\subsubsection{Third party libraries}
A third party library for rendering elements of the iPhone interface was employed.  While capable, it has not been maintained and suffers performance problems.  It is sufficient for prototype development, but if the system were to be adopted the iPhone interface styling and associated javascript would probably need to be improved.
\subsection{Constraints and validation}
The system is to be used by its owners, and for the sake of flexibility there are few constraints employed as to the nature of the data that may be entered.  
\subsubsection{Checking product policies during prescribing process}
One of the more significant potential uses of a system such as this, and a distinct advantage over paper-based equivalents, is the ability for local policies to be implemented with some level of automation.  As implemented this functionality is rudimentary.  Administrators are limited to pre-defined operations, whereas in order to obtain the maximum possible leverage over recorded information the set of possible conditional rules would need to be larger.
\subsection{Native application}
There are drawbacks in implementing the system as a web application.  While it increases the speed of development and improves accessibility to a variety of platforms, the browser is not as powerful or as fast as a local application.  It is likely that the web application implementation would be used as a prototype of the system and interface, and once tested and fine tuned, would act as a template for the implementation of a native application for the relevant platforms
\subsection{Separate system administration interface}
As it stands the system's administration is part of the standard interface, which is unnecessary.  A number of creation and update functions are not relevant in the day to day workings of a pharmacist on a ward.  For example, adding a new product or user administration would be best served by a different view, as would auditing functions, such as viewing all notes by all pharmacists.
\subsection{Performance}
Some of the information held in separate relations, such as in the status relation, could have been held as hashes instead.  However, doing so would have required changing the code to add new status types, whereas with a relation an administrator for the system can maintain status types.
\subsection{Logic}
The system is a structure into logic may be inserted, thanks largely to the MVC nature of rails applications.  Though some logic has been implemented, the specifics rely largely on the workings of a specific site.  Some sites would probably like the restrict dosing for a specific product to approved doses, while others may find that approach overly limiting.  The benefit of the restriction is that simply by not finding an odd dose would prompt the user to query the necessity for a non-standard regimen.
Due to the structure of the application, adding new logic is relatively straight forward - new methods can be defined in the model or controller and ActiveRecord. 
\bibliographystyle{plainurl}	% (uses file "plain.bst")
\bibliography{/Users/robert/Documents/Uni/Tex/myrefs}		% expects file "myrefs.bib"

\begin{comment}
    Appendices:
    Example of model code generation for pigyard

\end{comment}
\end{document}
